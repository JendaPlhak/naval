\documentclass[12pt,a4paper]{article}
\usepackage[utf8]{inputenc}
\usepackage[T1]{fontenc}
\usepackage[czech]{babel}
\usepackage{a4wide}
\usepackage{amsmath, amsthm, amsfonts, amssymb, graphicx, url, fancyhdr,multicol,enumerate,titling}
\usepackage{algpseudocode}
\usepackage[]{algorithm2e}
\usepackage{listings}
\theoremstyle{plain}
\newtheorem{thm}{Theorem} %[section]
\newtheorem{lemma}[thm]{Lemma}

\lstdefinelanguage{algpseudocode}{
  keywordstyle=[1]{\keywordstyle},
  keywordstyle=[2]{\operatorstyle},
  keywordstyle=[3]{\typestyle},
  keywordstyle=[4]{\functionstyle},
  identifierstyle={\identifierstyle},
  keywords=[1]{%
    begin,end,%
    program,procedure,function,subroutine,%
    while,do,for,to,next,repeat,until,loop,continue,endwhile,endfor,endloop,%
    if,then,else,endif,%
    return},
  literate={-}{$-$}1 {^}{$^\wedge$}1
           {>}{{$>$\ }}1 {<}{{$<$\ }}1
           {>=}{{$\geqslant$\ }}1 {<=}{{$\leqslant$\ }}1
           {:=}{{$\gets$\ }}1 {!=}{{$\ne$\ }}1 {<>}{{$\ne$\ }}1
           {->}{{$\;\to\;$}}1
           {&&}{{\keywordstyle and\ }}4 {{||}}{{\keywordstyle or\ }}3
           {;}{\hspace{0.2em};}2 {,}{\hspace{0.2em},}2,
}

\lstset{%
  language={algpseudocode},
  columns=fullflexible,
  numbers=left,
  numberstyle=\scriptsize,
}

\newcommand\keywordstyle{\rmfamily\bfseries\upshape}
\newcommand\operatorstyle{\rmfamily\mdseries\upshape}
\newcommand\typestyle{\rmfamily\mdseries\upshape}
\newcommand\functionstyle{\rmfamily\mdseries\scshape}

\newcommand\identifierstyle{\rmfamily\mdseries\itshape}

\newcommand\addkeywords[1]{%
  \lstset{morekeywords=[1]{#1}}}

\newcommand\addoperators[1]{%
  \lstset{morekeywords=[2]{#1}}}

\newcommand\addtypes[1]{%
  \lstset{morekeywords=[3]{#1}}}

\newcommand\addfunctions[1]{%
  \lstset{morekeywords=[4]{#1}}}

\newcommand{\Z}{\mathbb{Z}}
\newcommand{\N}{\mathbb{N}}

\begin{document}
\pagestyle{fancy}                      %Pro větší­ možnosti práce se záhlaví­mi a zápatími
\fancyhf{}                             %"vyčištění záhlaví a zápatí"
%\renewcommand{\headheight}{25 pt}                  %
\addtolength{\topmargin}{-30 pt}                   %
\setlength{\headsep}{10 pt}                      %
\fancyhead[L]{{\emph{IV003 - sada 1, příklad 3}}}  %
\fancyhead[R]{{\emph{Jan Plhák (UČO 408420), Vladimír Sedláček (UČO 408178)}}}                  % Nastavení­ pro titulní­ stranu
%\fancyfoot[L]{Školní rok 2009/2010}                %
%\renewcommand{\footrulewidth}{0.8 pt}              %
\renewcommand{\headrulewidth}{1 pt}                %               %

\addfunctions{PUSH}
\addfunctions{POP}
\addfunctions{QPUSH}
\addfunctions{QPULL}
\addfunctions{BPUSH}
\addfunctions{BPOP}
\addfunctions{BPULL}
\addfunctions{BALANCE}

V obou částech této úlohy si vždy budeme držet počítadla velikosti všech zásobníků, která vždy inkrementujeme, resp. dekrementujeme při volání funkcí PUSH, resp. POP. Budeme k tomu pouze konstantně mnoho paměti. Proto se níže budeme odkazovat na počet prvků v zásobníku, aniž bychom to explicitně zmiňovali. Asymptotická časová složitost tímto nijak ovlivněna nebude, protože bude amortizovaně konstantní, stejně jako všechny funkce, které navrhneme.

\begin{enumerate}[a)]
\item
Budeme uvažovat dva zásobníky $S_1,S_2$ (které jsou na začátku prázdné) spolu se standardními funkcemi PUSH a POP, kde PUSH($S_i,a$) vloží prvek $a$ na vrchol zásobníku $S_i$ a POP($S_i$) smaže vrchol zásobníku $S_i$. Frontu budeme implementovat pomocí funkcí QPUSH a QPULL, které definujeme následovně:

\begin{lstlisting}[mathescape]
function QPUSH($a$):
  PUSH($S_1,a$)
\end{lstlisting}

\begin{lstlisting}[mathescape]
function QPULL():
  if ($S_2=\emptyset$) 
    while ($S_1\neq \emptyset$)
         PUSH($S_2$,POP($S_1$))
  return POP($S_2$)
\end{lstlisting}

U funkce QPUSH je jasné, že její výpočet je vždy konečný. U funkce QPULL to plyne z toho, že cyklus while z řádků 3-4 proběhne právě tolikrát, kolik je počet prvků v $S_1$ před jeho zahájením, což je určitě konečné číslo.\\

Dále ukážeme, že QPUSH a QPULL skutečně implementují frontu. To plyne z následujících podmínek:
\begin{enumerate}[I.]
\item prvky v $S_1$ jsou od vrcholu po dno uspořádány vzestupně podle \uv{stáří}
\item prvky v $S_2$ jsou od vrcholu po dno uspořádány sestupně podle \uv{stáří}
\item libovolný prvek z $S_1$ je mladší než libovolný prvek z $S_2$
\item \uv{nejstarší} prvek je vždy buďto na vrcholu zásobníku $S_2$, resp. na dně zásobníku
 $S_1$, je-li $S_2$ prázdný.
\end{enumerate}
Na začátku jsou oba zásobníky prázdné, takže všechny podmínky triviálně platí. Platí-li tyto podmínky před zavoláním QPUSH, platí i po něm, neboť II,III a IV nejsou ovlivněny a \uv{nejmladší} prvek přijde na vrchol $S_1$. Dále předpokládejme, že všechny čtyři podmínky platí před zavoláním QPULL. V případě $S_2\neq\emptyset$ se podmínka I nemění, podmínka II zůstane zachována (podle IV jsme totiž smazali nejstarší prvek), podmínka III evidentně zůstane v platnosti (pouze jsme ji oslabili), a podmínka IV plyne ze II v případě, kdy v $S_2$ zbývají ještě nějaké další prvky, resp. z platnosti IV předtím v případě, že $S_2$ nyní bude prázdné. Konečně, v případě $S_2=\emptyset$ zůstanou po zavolání QPULL podmínky I a III v platnosti (neboť $S_1$ bude prázdný) a podmínky II a IV vyplynou z platnosti I před voláním QPULL.

Protože prvek, který \uv{ve finále} QPULL smaže, je vždy na vrcholu $S_2$, vyplývá z podmínek výše, že se skutečně jedná o implementaci fronty.\\

Nyní provedeme analýzu složitosti pomocí kreditové metody. Uvažme libovolnou posloupnost délky $n$ skládající se z funkcí QPUSH a QPULL. Operaci QPUSH přiřadíme čtyři kredity, operaci QPULL žádný. Skutečná cena QPUSH je přitom 1 a skutečná cena QPULL je 
$$\begin{cases}
1& \text{ pokud } |S_2|>0\\
2|S_1|& \text{ pokud } |S_2|=0.
\end{cases}$$
Ukážeme, že pro počet kreditů na účtu $k$ bude v každém okamžiku platit 
\begin{equation}\label{kr}
k=3|S_1|+|S_2|,
\end{equation} zejména tedy bude nezáporný. To zřejmě platí na začátku, kdy $k=0=|S_1|=|S_2|$. Použitím operace QPUSH se tato rovnost zachová, neboť pravá strana \eqref{kr} se zvýší o $(3(|S_1|+1)+|S_2|)-(3|S_1|+|S_2|)=3$ a levá taktéž o $4-1=3$. Podobně, při volání QPULL v případě $|S_2|>0$ se pravá strana \eqref{kr} sníží o $(3|S_1|)+|S_2|)-(3|S_1|+|S_2|-1)=1$ a levá strana se sníží o $1-0=1$. Konečně, při volání QPULL v případě $|S_2|=0$ se pravá strana \eqref{kr} sníží o $(3|S_1|)+0)-(3\cdot 0+|S_1|)=2|S_1|$ a levá se taktéž sníží o $2|S_1|-0=2|S_1|$. Tím jsme ukázali, že \eqref{kr} je invariantem naší struktury.\\

Protože QPUSH se v naší posloupnosti provede nejvýše $n$-krát, je celkový počet vytvořených kreditů nejvýše $4n$. Proto lze celkovou složitost celé posloupnosti odhadnout shora výrazem $4n\in\mathcal{O}(n)$. Z toho plyne, že amortizovaná složitost obou operací QPUSH i QPULL je konstantní.\\
\item
Uvažujme zásobníky $S_1,S_2,S_3$ se stejnými předpoklady jako v a). Datovou strukturu Both budeme implementovat pomocí funkcí BPUSH, BPOP a BPULL a pomocné funkce BALANCE, které definujeme níže:

\begin{lstlisting}[mathescape]
function BPUSH($a$):
  PUSH($S_1,a$)
\end{lstlisting}

\begin{lstlisting}[mathescape]
function BPOP():
  if ($S_1=\emptyset$) 
    BALANCE()
  return POP($S_1$)
\end{lstlisting}

\begin{lstlisting}[mathescape]
function BPULL():
  if ($S_2=\emptyset$)   
     while ($S_1\neq \emptyset$)
       PUSH($S_2$,POP($S_1$)) 
     BALANCE()
  return POP($S_2$)
\end{lstlisting}

\begin{lstlisting}[mathescape]
function BALANCE():
  for $i=1,\dots,\Big\lceil \frac{|S_2|}{2}\Big\rceil$
    PUSH($S_3$,POP($S_2$))   
  while ($S_2\neq \emptyset$)
       PUSH($S_1$,POP($S_2$))
  while ($S_3\neq \emptyset$)
       PUSH($S_2$,POP($S_3$))  
\end{lstlisting}
Konečnost výpočtu všech čtyř funkcí plyne z faktu, že všechny while cykly jsou vždy mají nejvýše tolik iterací, kolik je prvků v příslušném zásobníku $S_i$ (ze kterého zrovna odebíráme), což je určitě konečné číslo.\\

Dále ukážeme, že BPUSH, BPOP a QPULL skutečně implementují datovou strukturu Both. To plyne z následujících podmínek:
\begin{enumerate}[I.]
\item prvky v $S_1$ jsou od vrcholu po dno uspořádány vzestupně podle \uv{stáří}
\item prvky v $S_2$ jsou od vrcholu po dno uspořádány sestupně podle \uv{stáří}
\item libovolný prvek z $S_1$ je mladší než libovolný prvek z $S_2$
\item zásobník $S_3$ je před a po volání každé funkce prázdný
\item \uv{nejstarší} prvek je vždy buďto na vrcholu zásobníku $S_2$, resp. na dně zásobníku
 $S_1$, je-li $S_2$ prázdný.
\item \uv{nejmladší} prvek je vždy buďto na vrcholu zásobníku $S_1$, resp. na dně zásobníku
 $S_2$, je-li $S_1$ prázdný.
\end{enumerate}
Všimněme si, že platnost podmínek I-IV už zaručuje také platnost podmínek V a VI, proto stačí ukázat, že podmínky I-IV zůstanou všemi funkcemi BPUSH, BPOP a BPULL zachovány (na začátku tyto podmínky zřejmě platí). U podmínky IV je to jasné, protože do $S_3$ se prvky mohou dostat při volání funkce BALANCE, během které jsou však později všechny přesunuty zpět do $S_2$. Funkce BPUSH přidává nejmladší prvek na vrchol $S_1$, takže zachovává všechny podmínky. Nyní se zaměřme na funkci BPOP. V případě, kdy $S_1\neq\emptyset$, pouze smaže vrchní prvek, takže podmínky I-III zřejmě zůstanou zachovány. V opačném případě máme $S_2\neq\emptyset$ (jinak jsou všechny zásobníky prázdné a vše triviálně platí), takže po zavolání BALANCE se spodní polovina prvků z $S_2$ ocitne v $S_1$ v opačném pořadí a vrchní polovina prvků zůstane v $S_2$ ve stejném pořadí. Díky platnosti podmínky II před voláním funkce a faktu, že $S_1$ byl prázdný, tak snadno vidíme, že všechny tři podmínky I-III zůstanou v plastnosti, a to i po následném POP($S_1$). U funkce BPULL můžeme argumentovat zcela analogicky, pouze je potřeba si uvědomit, že v případě $S_2=\emptyset$ se díky podmínce I zachovají podmínky I-III (a finální POP($S_2$) na konci BPULL také nic nepokazí). Tím jsme ukázali, že podmínky V a VI platí po celou dobu, z čehož plyne korektnost celého algoritmu (stačí se podívat, že odpovídající funkce mažou prvky ze správných míst).

Teď provedeme analýzu časové složitosti. Cena BPUSHe je zřejmě 1. Dále je vidět, že každé volání funkce BALANCE nás stojí $$2\lceil\frac{|S_2|}{2}\rceil+2(|S_2|-\lceil\frac{|S_2|}{2}\rceil)+2\lceil\frac{|S_2|}{2}\rceil\leq 3|S_2|$$
časových jednotek, z čeho snadno plyne, že cena BPOPu je
$$\begin{cases}
1& \text{ pokud } |S_1|>0\\
3|S_2|+1& \text{ pokud } |S_1|=0
\end{cases}$$
a  cena BPULLu je
$$\begin{cases}
1& \text{ pokud } |S_2|>0\\
5|S_1|+1& \text{ pokud } |S_2|=0.
\end{cases}$$
Nyní si stačí uvědomit, že po každém \uv{drahém} volání BPOPu s $|S_2|=k>0$, resp. BPULLu s $|S_1|=k>0$ bude následujících nejméně $\lceil \frac{k}{2}\rceil$ operací \uv{levných} (neboť všechny tři funkce budou za těchto podmínek mít cenu 1) tj. celková cena takovéhoto úseku $\lceil \frac{k}{2}\rceil+1$ operací bude $3k+1+k\in\mathcal{O}(k)$, resp. $5k+1+k\in\mathcal{O}(k)$. Protože zároveň $\lceil \frac{k}{2}\rceil+1\in \mathcal{O}(k)$, můžeme tvrdit, že po \uv{zprůměrování} budou všechny tři funkce amortizovaně konstantní (přesněji, pokud bychom si libovolnou posloupnost našich tří funkcí délky $n$ rozdělili na úseky tohoto typu a úseky obsahující pouze BPUSH, můžeme z argumentace výše vidět, že celková časová složitost této posloupnosti bude v $\mathcal{O}(n)$, s možnou výjimkou jediného neukončeného úseku na konci naší posloupnosti, který ale asymptotickou složitost nezmění.)

(Pro ještě preciznější argumentaci by bylo potřeba použít potenciálovou funkci, ale volba $\Phi(i):=a|S_1|(i)+b|S_2|(i)$ pro $a,b\in\mathbb{R}$, kde $|S_j|(i)$ značí počet prvků $j$-tého zásobníku po $i$-tém kroku, vedla na systém dvou nerovností $|S_2|(6+a-b)\leq 0$, $|S_1|(10+b-a)\leq 0$, který nemá řešení. Možná se někde stala chyba při výpočtu.)
\end{enumerate}
\end{document}