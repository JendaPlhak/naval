\documentclass[12pt,a4paper]{article}
\usepackage[utf8]{inputenc}
\usepackage[T1]{fontenc}
\usepackage[czech]{babel}
\usepackage{a4wide}
\usepackage{amsmath, amsthm, amsfonts, amssymb, graphicx, url, fancyhdr,multicol,enumerate,titling}
\usepackage{algpseudocode}
\usepackage[]{algorithm2e}
\usepackage{listings}
\theoremstyle{plain}
\newtheorem{thm}{Theorem} %[section]
\newtheorem{lemma}[thm]{Lemma}

\lstdefinelanguage{algpseudocode}{
  keywordstyle=[1]{\keywordstyle},
  keywordstyle=[2]{\operatorstyle},
  keywordstyle=[3]{\typestyle},
  keywordstyle=[4]{\functionstyle},
  identifierstyle={\identifierstyle},
  keywords=[1]{%
    begin,end,%
    program,procedure,function,subroutine,%
    while,do,for,to,next,repeat,until,loop,continue,endwhile,endfor,endloop,%
    if,then,else,endif,%
    return},
  literate={-}{$-$}1 {^}{$^\wedge$}1
           {>}{{$>$\ }}1 {<}{{$<$\ }}1
           {>=}{{$\geqslant$\ }}1 {<=}{{$\leqslant$\ }}1
           {:=}{{$\gets$\ }}1 {!=}{{$\ne$\ }}1 {<>}{{$\ne$\ }}1
           {->}{{$\;\to\;$}}1
           {&&}{{\keywordstyle and\ }}4 {{||}}{{\keywordstyle or\ }}3
           {;}{\hspace{0.2em};}2 {,}{\hspace{0.2em},}2,
}

\lstset{%
  language={algpseudocode},
  columns=fullflexible,
  numbers=left,
  numberstyle=\scriptsize,
}

\newcommand\keywordstyle{\rmfamily\bfseries\upshape}
\newcommand\operatorstyle{\rmfamily\mdseries\upshape}
\newcommand\typestyle{\rmfamily\mdseries\upshape}
\newcommand\functionstyle{\rmfamily\mdseries\scshape}

\newcommand\identifierstyle{\rmfamily\mdseries\itshape}

\newcommand\addkeywords[1]{%
  \lstset{morekeywords=[1]{#1}}}

\newcommand\addoperators[1]{%
  \lstset{morekeywords=[2]{#1}}}

\newcommand\addtypes[1]{%
  \lstset{morekeywords=[3]{#1}}}

\newcommand\addfunctions[1]{%
  \lstset{morekeywords=[4]{#1}}}

\newcommand{\Z}{\mathbb{Z}}
\newcommand{\N}{\mathbb{N}}

\begin{document}
\pagestyle{fancy}                      %Pro větší­ možnosti práce se záhlaví­mi a zápatími
\fancyhf{}                             %"vyčištění záhlaví a zápatí"
%\renewcommand{\headheight}{25 pt}                  %
\addtolength{\topmargin}{-30 pt}                   %
\setlength{\headsep}{10 pt}                      %
\fancyhead[L]{{\emph{IV003 - sada 1, příklad 2}}}  %
\fancyhead[R]{{\emph{Jan Plhák (UČO 408420), Vladimír Sedláček (UČO 408178)}}}                  % Nastavení­ pro titulní­ stranu
%\fancyfoot[L]{Školní rok 2009/2010}                %
%\renewcommand{\footrulewidth}{0.8 pt}              %
\renewcommand{\headrulewidth}{1 pt}                %               %

\addfunctions{VYHLED}
\addfunctions{PUSH}
\addfunctions{POP}

Nejprve ukážeme, že je VYHLED korektní. Konečnost výpočtu plyne z faktu, že PUSH se provede právě $n$-krát, tudíž POP nejvýše $n$-krát.\\

V následujícím textu budeme často hovořit o "prvcích zásobníku", přičemž tímto obratem budeme mínit mrakodrap, který je tímto prvkem - indexem jednoznačně určen. Podobně budeme používat "mrakodrap ze zásobníku". Nyní ukážeme, že pro každé $k\in\{0,1,\dots,n\}$ je pravda, že po $k$-té iteraci cyklu for platí tvrzení

\begin{equation} \label{invariant}
\forall i\in \N, 1\leq i\leq k: A[i]\in S \Leftrightarrow (\forall j\in\N, i<j\leq k: A[i]>A[j]).
\end{equation}

To je zřejmě pravda pro $k=0$, resp. $k=1$ (v těchto případech je zásobník prázdný, resp. je v něm právě prvek $A[1]$). Stačí tedy ukázat, že z pravdivosti tohoto faktu pro $k$ vyplývá i jeho pravdivost pro $k+1$ (potom totiž bude platit i pro $n$ a formule výše nám zajistí korektnost celého algoritmu). Nejprve poznamenejme, že z indukčního předpokladu plyne, že prvky zásobníku (mrakodrapy, na které prvky ukazují) tvoří klesající posloupnost (ode dna po vrchol). Z toho plyne, že while cyklus na řádcích 3-5 odstraní ze zásobníku všechny mrakodrapy, které nevidí přes prvek $A[k+1]$. 

Nyní vyšetříme platnost ekvivalence \eqref{invariant} po skončení kroku $k+1$  v následujících případech:

\begin{itemize}
\item $i = k + 1$:\\
Podmínka je triviálně splněna neboť prvek $A[k+1]$ byl do $S$ přidán na řádku 8 v aktuálním kroku $k+1$.

\item $1 \leq i < k + 1$ implikace "$ \Leftarrow $":\\
Obměnou implikace dostáváme: 
\begin{equation} \label{alter}
A[i]\not\in S \Rightarrow (\exists j\in\N, i<j\leq k + 1: A[i] \leq A[j]).
\end{equation}
Tuto implikaci ukážeme následujícím způsobem: mějme $ A[i] \not\in S$ z předpokladu implikace. Pak protože $ i < k + 1 $, museli jsme prvek $ A[i] $ v předchozím kroku $ i $ přidat do zásobníku (řádek 8). Jediný způsob jakým jsme tento prvek mohli ze zásobníku zase odebrat je, že v nějakém z pozdějších kroků (označme jej $ j $) platilo $ A[j] \geq A[TOP(S)]$ a zároveň $ TOP(S) = i $. $ j $ je tedy hledaný index a implikace je dokázána.

\item $1 \leq i < k + 1$ implikace "$ \Rightarrow $":\\
Nechť naopak $ A[i] \in S $. Pak ovšem také $ A[i] \in S_k $ a z indukčního předpokladu plyne, že 
\begin{equation}
A[i]\in S \Rightarrow (\forall j\in\N, i < j < k + 1: A[i]>A[j]).
\end{equation}
Zbývá ukázat nerovnost $ A[i] > A[k + 1] $. Ta ale platí díky tomu, že mrakodrapy v zásobníku tvoří klesající posloupnost a speciálně $ A[TOP(S)] > A[k + 1] $ (jinak bychom dostali spor s podmínkou while cyklu a toho, že cyklus skončil.). Celkem $ A[i] \geq A[TOP(S)] > A[k + 1] $ a důkaz je hotov.

\end{itemize}


Nyní provedeme analýzu složitosti funkce VYHLED pomocí kreditové metody. Operaci PUSH přiřadíme dva kredity, operaci POP žádný (skutečná cena obou dvou je přitom~1). Potom je zřejmé, že počet kreditů na účtu bude v každém okamžiku roven počtu prvků v zásobníku $S$, zejména tedy bude nezáporný. Přitom během volání funkce výhled se operace PUSH provede právě $n$-krát, tudíž celkový počet vytvořených kreditů je $2n$. Proto je celková složitost v nejhorším případě rovna $2n\in\mathcal{O}(n)$.\\

Naopak, na každý mrakodrap se určitě musíme alespoň jednou podívat, protože mezi výškami mrakodrapů není a priori žádný vztah. Mrakodrapů je však alespoň $n$, takže asymptotická složitost VYHLEDu je v $\Omega(n)$. Celkově je tedy asymptotická složitost VYHLEDu v $\Theta(n)$.

\end{document}
