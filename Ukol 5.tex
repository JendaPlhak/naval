\documentclass[12pt,a4paper]{article}
\usepackage[utf8]{inputenc}
\usepackage[T1]{fontenc}
\usepackage[czech]{babel}
\usepackage{a4wide}
\usepackage{amsmath, amsthm, amsfonts, amssymb, graphicx, url, fancyhdr,multicol,enumerate,titling}
\usepackage{algpseudocode}
\usepackage[]{algorithm2e}
\usepackage{listings}
\theoremstyle{plain}
\newtheorem{thm}{Theorem} %[section]
\newtheorem{lemma}[thm]{Lemma}

\lstdefinelanguage{algpseudocode}{
  keywordstyle=[1]{\keywordstyle},
  keywordstyle=[2]{\operatorstyle},
  keywordstyle=[3]{\typestyle},
  keywordstyle=[4]{\functionstyle},
  identifierstyle={\identifierstyle},
  keywords=[1]{%
    begin,end,%
    program,procedure,function,subroutine,%
    while,do,for,to,next,repeat,until,loop,continue,endwhile,endfor,endloop,%
    if,then,else,endif,%
    return},
  literate={-}{$-$}1 {^}{$^\wedge$}1
           {>}{{$>$\ }}1 {<}{{$<$\ }}1
           {>=}{{$\geqslant$\ }}1 {<=}{{$\leqslant$\ }}1 
           {:=}{{$\gets$\ }}1 {!=}{{$\ne$\ }}1 {<>}{{$\ne$\ }}1
           {->}{{$\;\to\;$}}1
           {&&}{{\keywordstyle and\ }}4 {{||}}{{\keywordstyle or\ }}3
           {;}{\hspace{0.2em};}2 {,}{\hspace{0.2em},}2,
}

\lstset{%
  language={algpseudocode},
  columns=fullflexible,
  numbers=left,
  numberstyle=\scriptsize,
}

\newcommand\keywordstyle{\rmfamily\bfseries\upshape}
\newcommand\operatorstyle{\rmfamily\mdseries\upshape}
\newcommand\typestyle{\rmfamily\mdseries\upshape}
\newcommand\functionstyle{\rmfamily\mdseries\scshape}

\newcommand\identifierstyle{\rmfamily\mdseries\itshape}

\newcommand\addkeywords[1]{%
  \lstset{morekeywords=[1]{#1}}}

\newcommand\addoperators[1]{%
  \lstset{morekeywords=[2]{#1}}}

\newcommand\addtypes[1]{%
  \lstset{morekeywords=[3]{#1}}}

\newcommand\addfunctions[1]{%
  \lstset{morekeywords=[4]{#1}}}

\newcommand{\Z}{\mathbb{Z}}
\newcommand{\N}{\mathbb{N}}

\begin{document}
\pagestyle{fancy}                      %Pro větší­ možnosti práce se záhlaví­mi a zápatími
\fancyhf{}                             %"vyčištění záhlaví a zápatí"                                         
%\renewcommand{\headheight}{25 pt}                  %
\addtolength{\topmargin}{-30 pt}                   %
\setlength{\headsep}{10 pt}                      %
\fancyhead[L]{{\emph{IV003 - sada 1, příklad 5}}}  %
\fancyhead[R]{{\emph{Jan Plhák (UČO 408420), Vladimír Sedláček (UČO 408178)}}}                  % Nastavení­ pro titulní­ stranu
%\fancyfoot[L]{Školní rok 2009/2010}                %
%\renewcommand{\footrulewidth}{0.8 pt}              %
\renewcommand{\headrulewidth}{1 pt}                %               %

\addfunctions{NEJVYSSI\_VPRAVO(l)}
\addfunctions{NEJVYSSI\_NAD(l)}

Datová struktura se bude skládat ze dvou polí $X,Y$, přičemž prvky $X$ budou uspořádané čtveřice $(a,b,c,d)$, kde $(c,d)$ je vždy nejvyšší bod napravo od $(a,b)$, tj. $$(c,d)=\sup\{(x,y)\in S | x\geq a\}, $$
přičemž supremum bereme vzhledem k uspořádání $\preceq_Y\subseteq S\times S$ definovaného vztahem 
\begin{equation}\label{1}
(s,t)\preceq_Y (u,v) \Longleftrightarrow (t<v \vee (t=v \wedge s\leq u)).
\end{equation}
Navíc budeme požadovat, aby posloupnost prvních složek $X$ byla neklesající.

Analogicky, prvky $Y$ budou uspořádané čtveřice $(a,b,c,d)$, kde $(c,d)$ je vždy nejpravější bod nad $(a,b)$, tj. $$(c,d)=\sup\{(x,y)\in S | y\geq b\},$$
přičemž supremum bereme vzhledem k uspořádání $\preceq_X\subseteq S\times S$ definovaného vztahem
\begin{equation}\label{2}
(s,t)\preceq_X (u,v) \Longleftrightarrow (s<u \vee (s=u \wedge t\leq v)).
\end{equation}
Navíc budeme požadovat, aby posloupnost druhých složek $Y$ byla neklesající.

Je jasné, že paměťová složitost obou těchto polí dohromady bude $2\cdot 2|S|=4n\in\Theta(n)$. Na časovou složitost jejich vytvoření přitom v zadání nebyly kladeny žádné požadavky, proto nejprve naivně vytvoříme příslušné čtveřice tak, že pro každý bod $(a,b)\in S$ projdeme všechny prvky $S$ a vybereme příslušný bod $(c,d)$ splňující podmínku \eqref{1}, resp. \eqref{2} (pro pole $X$, resp. $Y$). Následně vzniklá pole čtveřic $(a,b,c,d)$ vzestupně uspořádáme podle prvních, resp. druhých složek.

Funkce definujeme následovně:

\begin{lstlisting}
function NEJVYSSI_VPRAVO(l)
  r := a mod b
  while r != 0 do
    a := b
    b := r
    r := a mod b
  end while
  return HEAPIFY b
end function
\end{lstlisting}

\begin{lstlisting}
function NEJVYSSI_VPRAVO(l)
  r := a mod b
  while r != 0 do
    a := b
    b := r
    r := a mod b
  end while
  return b
end function
\end{lstlisting}


\begin{lemma} NEJVYSSI\_VPRAVO(l) je korektní a zachovává všechny invarianty.
\end{lemma}
\begin{proof}
Korektnost plyne z toho, že jsme do jedné z hald přidali nový prvek a po konečně mnoha vý.
\end{proof}

\end{document}