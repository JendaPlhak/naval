\documentclass[12pt,a4paper]{article}
\usepackage[utf8]{inputenc}
\usepackage[T1]{fontenc}
\usepackage[czech]{babel}
\usepackage{a4wide}
\usepackage{amsmath, amsthm, amsfonts, amssymb, graphicx, url, fancyhdr,multicol,enumerate,titling}
\usepackage{algpseudocode}
\usepackage[]{algorithm2e}
\usepackage{listings}
\theoremstyle{plain}
\newtheorem{thm}{Theorem} %[section]
\newtheorem{lemma}[thm]{Lemma}

\lstdefinelanguage{algpseudocode}{
  keywordstyle=[1]{\keywordstyle},
  keywordstyle=[2]{\operatorstyle},
  keywordstyle=[3]{\typestyle},
  keywordstyle=[4]{\functionstyle},
  identifierstyle={\identifierstyle},
  keywords=[1]{%
    begin,end,%
    program,procedure,function,subroutine,%
    while,do,for,to,next,repeat,until,loop,continue,endwhile,endfor,endloop,%
    if,then,else,endif,%
    return},
  literate={-}{$-$}1 {^}{$^\wedge$}1
           {>}{{$>$\ }}1 {<}{{$<$\ }}1
           {>=}{{$\geqslant$\ }}1 {<=}{{$\leqslant$\ }}1 
           {:=}{{$\gets$\ }}1 {!=}{{$\ne$\ }}1 {<>}{{$\ne$\ }}1
           {->}{{$\;\to\;$}}1
           {&&}{{\keywordstyle and\ }}4 {{||}}{{\keywordstyle or\ }}3
           {;}{\hspace{0.2em};}2 {,}{\hspace{0.2em},}2,
}

\lstset{%
  language={algpseudocode},
  columns=fullflexible,
  numbers=left,
  numberstyle=\scriptsize,
}

\newcommand\keywordstyle{\rmfamily\bfseries\upshape}
\newcommand\operatorstyle{\rmfamily\mdseries\upshape}
\newcommand\typestyle{\rmfamily\mdseries\upshape}
\newcommand\functionstyle{\rmfamily\mdseries\scshape}

\newcommand\identifierstyle{\rmfamily\mdseries\itshape}

\newcommand\addkeywords[1]{%
  \lstset{morekeywords=[1]{#1}}}

\newcommand\addoperators[1]{%
  \lstset{morekeywords=[2]{#1}}}

\newcommand\addtypes[1]{%
  \lstset{morekeywords=[3]{#1}}}

\newcommand\addfunctions[1]{%
  \lstset{morekeywords=[4]{#1}}}

\newcommand{\Z}{\mathbb{Z}}
\newcommand{\N}{\mathbb{N}}

\begin{document}
\pagestyle{fancy}                      %Pro větší­ možnosti práce se záhlaví­mi a zápatími
\fancyhf{}                             %"vyčištění záhlaví a zápatí"                                         
%\renewcommand{\headheight}{25 pt}                  %
\addtolength{\topmargin}{-30 pt}                   %
\setlength{\headsep}{10 pt}                      %
\fancyhead[L]{{\emph{IV003 - sada 1, příklad 4}}}  %
\fancyhead[R]{{\emph{Jan Plhák (UČO 408420), Vladimír Sedláček (UČO 408178)}}}                  % Nastavení­ pro titulní­ stranu
%\fancyfoot[L]{Školní rok 2009/2010}                %
%\renewcommand{\footrulewidth}{0.8 pt}              %
\renewcommand{\headrulewidth}{1 pt}                %               %

\addfunctions{INSERT}
\addfunctions{DELETE_MIN}
\addfunctions{DELETE_MAX}
\addfunctions{HEAPIFY}
\addfunctions{heapify_up}

Použijeme binární minimovou haldu $A$ a binární maximovou haldu $B$. Tyto haldy budeme reprezentovat dvěma poli $[A_0,A_1,\dots,A_k]$, resp. $[B_0,B_1,\dots,B_l]$ stadardním způsobem, tj. tak, aby platilo $$A_i\leq A_{2i+1}, A_i\leq A_{2i+2}, B_j\geq B_{2j+1}, B_j\geq B_{2j+2}$$ pro všechna $i\in \N_0$ a $j\in \N_0$, pro která je alespoň jeden z příslušných prvků definován. Navíc budeme požadovat, aby po zavolání každé z funkcí INSERT, DELETE\_MIN, DELETE\_MAX platilo $l\leq k \leq l+1$ a zároveň $A_i\leq B_i$ pro všechna $i\in \{0,1,\dots,l\}$. (Iniciálně to triviálně platí.)\\
Funkce definujeme následovně:

\begin{lstlisting}[mathescape]
function min(T)
  if |A| = 0
    return Error
  else:
    return $A_0$
end function
\end{lstlisting}

\begin{lstlisting}[mathescape]
function max(T)
  if $|A| + |B| = 0$
    return Error
  else if $|B| = 0$
    return $ A_0 $
  else 
    return $B_0$
end function
\end{lstlisting}

\begin{lstlisting}[mathescape]
function INSERT(x,T)
  if $|B| \not= 0 \wedge max(T) < x$
    swap($B_0$, x)
    return INSERT(x,T)
  
  if |A| = |B|
    append(x, A)
    heapify_up(k+1, A)
  else
    append(x, B)
    heapify_up(l+1, B)
    if $A_{l+1} > B_{l+1} $
      swap($A_{l+1}, B_{l+1}$)
      heapify_up(l+1, A)
      heapify_up(l+1, B)
end function
\end{lstlisting}

\begin{lstlisting}
function DELETE_MIN(T) 
  r := a mod b
  while r != 0 do
    a := b
    b := r
    r := a mod b
  end while
  return b
end function
\end{lstlisting}

\begin{lstlisting}
function DELETE_MAX(T)
  r := a mod b
  while r != 0 do
    a := b
    b := r
    r := a mod b
  end while
  return b
end function
\end{lstlisting}

\begin{lemma} INSERT je korektní a zachovává všechny invarianty.
\end{lemma}
\begin{proof}
Korektnost plyne z toho, že jsme do jedné z hald přidali nový prvek a po konečně mnoha vý.
\end{proof}

\end{document}