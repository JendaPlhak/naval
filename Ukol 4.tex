\documentclass[12pt,a4paper]{article}
\usepackage[utf8]{inputenc}
\usepackage[T1]{fontenc}
\usepackage[czech]{babel}
\usepackage{a4wide}
\usepackage{amsmath, amsthm, amsfonts, amssymb, graphicx, url, fancyhdr,multicol,enumerate,titling}
\usepackage{algpseudocode}
\usepackage[]{algorithm2e}
\usepackage{listings}
\theoremstyle{plain}
\newtheorem{thm}{Theorem} %[section]
\newtheorem{lemma}[thm]{Lemma}

\lstdefinelanguage{algpseudocode}{
  keywordstyle=[1]{\keywordstyle},
  keywordstyle=[2]{\operatorstyle},
  keywordstyle=[3]{\typestyle},
  keywordstyle=[4]{\functionstyle},
  identifierstyle={\identifierstyle},
  keywords=[1]{%
    begin,end,%
    program,procedure,function,subroutine,%
    while,do,for,to,next,repeat,until,loop,continue,endwhile,endfor,endloop,%
    if,then,else,endif,%
    return},
  literate={-}{$-$}1 {^}{$^\wedge$}1
           {>}{{$>$\ }}1 {<}{{$<$\ }}1
           {>=}{{$\geqslant$\ }}1 {<=}{{$\leqslant$\ }}1 
           {:=}{{$\gets$\ }}1 {!=}{{$\ne$\ }}1 {<>}{{$\ne$\ }}1
           {->}{{$\;\to\;$}}1
           {&&}{{\keywordstyle and\ }}4 {{||}}{{\keywordstyle or\ }}3
           {;}{\hspace{0.2em};}2 {,}{\hspace{0.2em},}2,
}

\lstset{%
  language={algpseudocode},
  columns=fullflexible,
  numbers=left,
  numberstyle=\scriptsize,
}

\newcommand\keywordstyle{\rmfamily\bfseries\upshape}
\newcommand\operatorstyle{\rmfamily\mdseries\upshape}
\newcommand\typestyle{\rmfamily\mdseries\upshape}
\newcommand\functionstyle{\rmfamily\mdseries\scshape}

\newcommand\identifierstyle{\rmfamily\mdseries\itshape}

\newcommand\addkeywords[1]{%
  \lstset{morekeywords=[1]{#1}}}

\newcommand\addoperators[1]{%
  \lstset{morekeywords=[2]{#1}}}

\newcommand\addtypes[1]{%
  \lstset{morekeywords=[3]{#1}}}

\newcommand\addfunctions[1]{%
  \lstset{morekeywords=[4]{#1}}}

\newcommand{\Z}{\mathbb{Z}}
\newcommand{\N}{\mathbb{N}}

\begin{document}
\pagestyle{fancy}                      %Pro větší­ možnosti práce se záhlaví­mi a zápatími
\fancyhf{}                             %"vyčištění záhlaví a zápatí"                                         
%\renewcommand{\headheight}{25 pt}                  %
\addtolength{\topmargin}{-30 pt}                   %
\setlength{\headsep}{10 pt}                      %
\fancyhead[L]{{\emph{IV003 - sada 1, příklad 4}}}  %
\fancyhead[R]{{\emph{Jan Plhák (UČO 408420), Vladimír Sedláček (UČO 408178)}}}                  % Nastavení­ pro titulní­ stranu
%\fancyfoot[L]{Školní rok 2009/2010}                %
%\renewcommand{\footrulewidth}{0.8 pt}              %
\renewcommand{\headrulewidth}{1 pt}                %               %

\addfunctions{INSERT}
\addfunctions{DELETE_MIN}
\addfunctions{DELETE_MAX}
\addfunctions{HEAPIFY}
\addfunctions{heapify_up}

Použijeme binární minimovou haldu $A$ a binární maximovou haldu $B$. Tyto haldy budeme reprezentovat dvěma poli $[A_0,A_1,\dots,A_k]$, resp. $[B_0,B_1,\dots,B_l]$ stadardním způsobem, tj. tak, aby platilo $$A_i\leq A_{2i+1}, A_i\leq A_{2i+2}, B_j\geq B_{2j+1}, B_j\geq B_{2j+2}$$ pro všechna $i\in \N_0$ a $j\in \N_0$, pro která je alespoň jeden z příslušných prvků definován. Navíc budeme požadovat, aby po zavolání každé z funkcí INSERT, DELETE\_MIN, DELETE\_MAX platilo (invariant):
\begin{enumerate}
\item $l\leq k \leq l+1$
\item $A_i\leq B_i$ pro všechna $i\in \{0,1,\dots,l\}$. (Iniciálně to triviálně platí.)
\item Funkce min resp. max korektně vrací min. resp. max prvek struktury. 
\item A je minimová halda
\item B je maximová halda
\end{enumerate}

Dále budeme používat standardně definovanou funkci heapify\_up, která zajistí probublání prvku haldou odspoda nahoru vzhledem k uspořádání minimové hlady A nebo maximové hlady B. \\
Funkce definujeme následovně:

\begin{lstlisting}[mathescape]
function INSERT(x,T)
  if $B \not= \emptyset \wedge max(T) < x$
    swap($B_0$, x)
    return INSERT(x,T)
  
  if |A| = |B|
    append(x, A)
    heapify_up(k+1, A)
  else
    append(x, B)
    heapify_up(l+1, B)
    if $A_{l+1} > B_{l+1} $
      swap($A_{l+1}, B_{l+1}$)
      heapify_up(l+1, A)
      heapify_up(l+1, B)
end function
\end{lstlisting}

\begin{lstlisting}[mathescape]
function DELETE_MIN(T) 
  if $A = \emptyset$
    return Error
  if $ |A| = 1 \wedge |B| = 0 $
    return pop_tail(T)
  min := $ A_0 $
  $ A_0 $ := pop_tail(T) 
  heapify_down($(A, <)$, 0)
  return min
end function
\end{lstlisting}

\begin{lstlisting}[mathescape]
function DELETE_MAX(T) 
  if $A = \emptyset$
    return Error
  if $ |A| = 1 \vee |B| = 1 $
    return pop_tail(T)
  max := $ B_0 $
  $ B_0 $ := pop_tail(T) 
  heapify_down($(B, >)$, 0)
  return max
end function
\end{lstlisting}

\begin{lstlisting}[mathescape]
function heapify_down($(Heap, \succ), i$)
  (left, right) = children($Heap_{i}$)
  if (left = Nil)
    balance_up(idx)
  else if (right = Nil)
    if ($Heap_{left} \succ Heap_i $)
      swap($Heap_{left}, Heap_i$)
    balance_up(left)
  else 
    X := $\{  d \mid d \in \{Heap_{left}, Heap_{right}\} \wedge d \succ Heap_{i} \} $
    if ($X \not = \emptyset$) 
      dir := index of greater element (with respect $\text{to} \succ$) from X.
      swap($ Heap_{dir}, Heap_i$)
      heapif	y_down($(Heap, \succ), dir$)
end function
\end{lstlisting}

\begin{lstlisting}[mathescape]
function balance_up(i)
  if $i \geq |B|$
    return
  if $A_i > B_i$
    swap($A_i, B_i$)
    heapify_up(i, A)
    heapify_up(i, B)
end function
\end{lstlisting}

\begin{lstlisting}[mathescape]
function min(T)
  if |A| = 0
    return Error
  else:
    return $A_0$
end function
\end{lstlisting}

\begin{lstlisting}[mathescape]
function max(T)
  if $|A| + |B| = 0$
    return Error
  else if $|B| = 0$
    return $ A_0 $
  else 
    return $B_0$
end function
\end{lstlisting}

\begin{lstlisting}[mathescape]
function pop_tail(T)
  if $|A| = |B|$
    return pop_back(B)
  else 
    return pop_back(A)
end function
\end{lstlisting}

\begin{lemma} INSERT je korektní a zachovává všechny invarianty.
\end{lemma}
\begin{proof}
Neboť funkce INSERT neobsahuje žádné cykly ani rekurzivní volání a o funkci heapify\_up můžeme předpokládat, že je korektní, je zřejmé, že výpočet bude vždy konečný.\\

Pokud dojde ke splnění podmínky na řádku 2, je z invariantu zřejmé, že se jedná o nové maximum a jeho vložením na pozici $ B_0 $ zůstane invariant zachován. Při rekurzivním volání již podmínka na řádku 2 nemůže být splněna a korektnost této větve je závislá na korektnosti řádků 6-15. \\

Nechť tedy není splněna podmínka na řádku 2. Vyšetříme dva případy:
\begin{itemize}
\item $|A| = |B| $:\\
Po provedení řádků 7 a 8 řejmě zůstane v platnosti podmínka 1 a také 5 protože haldu B nijak nemodifikujeme. Aplikace funkce heapify\_up zajistí, že A bude po přidání prvku x opět minimovou haldou, tedy splnění podmínky 4. Pruchod funkce haldou A na řádku 8 jednoznačně určuje cestu haldou. Tuto cestu označme [$A_0, \dots, A_{i-1}, x, A_i, \dots, A_k $] a odpovídající cestu [$B_0, \dots, B_{i-1}, B_i, B_{i + 1}, \dots, B_k $]. Pak zřejmě $ A_j \leq B_j \forall j \in \N, 0 \leq j < i $, neboť tyto prvky nebyly změněny. Dále $ x \leq A_i \leq B_i $ a analogicky pro všechny prvky pod $ x $, až na $ A_k $, pro které ale splnění podmínky 2 v nové haldě nepožadujeme (nemá odpovídající protějšek v B). Tím je dokázána platnost podmínky 2. Konečně neplatnost podmínky na řádku 2 zajišťuje, že max bude korektní a z vlastností heapify\_up pro minimovou haldu vyplývá, že min také.


\item $|A| = |B| + 1 $ (z podmínky 1 plyne, že jiná možnost není):\\
Řádky 10-11 zajistí splnění podmínek 1, 4, 5 a 2 až na $A_{l+1} \leq B_{l+1} $ díky analogické argumentaci jako v předchozím případě. Pokud tedy platí $A_{l+1} \leq B_{l+1} $, jsme hotovi. V opačném případě prvky prohodíme, čímž zajistíme splnění podmínky 2, ale potenciálně pokazíme 4 a 5. Tento problém opravíme na řádcích 14-15. \\

Zbývá si rozmyslet, že jsme tím nepokazili podmínku 2. To ale můžeme udělat stejně jako v předchozím bodě, avšak navíc je třeba nejprve odargumentovat poslední prvek (nejspodnější) posloupnosti určené funkcí heapify\_up, neboť nyní po vložení platí $ |A| = |B| $. Označme $A' $ a $B'$ posloupnosti určené průchodem funkce heapify\_up a $i_l $ index posledního prvku těchto posloupností. Pak musela nastat jedna z následujících možností:

\begin{itemize}
\item $A'_{i_l} = B_{i_l} \wedge B'_{i_l} = A_{i_l} $: \\
V tomto případě neměly řádky 14 a 15 žádný efekt a podmínka 2 tak triviálně platí.

\item $A'_{i_l} = B_{i_l} \wedge B'_{i_l} = B_{i_{l - 1}} $: \\
řádek 14 nic, ale v 15 došlo k propagaci. Pak ale z invariantu (5) $ B_{i_l} \leq B_{i_{l - 1}} $.
\item $A'_{i_l} = A_{i_{l - 1}} \wedge B'_{i_l} = A_{i_l} $: \\
řádek 15 nic, ale v 14 došlo k propagaci. Analogicky jako v předchozím případě $ A_{i_{l - 1}} \leq A_{i_l} $.

\item $A'_{i_l} = A_{i_{l - 1}} \wedge B'_{i_l} = B_{i_{l - 1}} $: \\
řádek 14 propagace, 15 propagace. Platnost 2 pro $A'$ a $B'$ triviálně z platnosti 2 pro $A$ a $B$.

\end{itemize}

Konečně z vlastností minimové a maximové haldy a neplatnosti podmínky 2 vyplývá platnost podmínky 3.
\end{itemize}
\end{proof}

\begin{lemma} DELETE\_MIN je korektní a zachovává všechny invarianty.
\end{lemma}
\begin{proof}
Snadno se nahlédne, že rekurze bude konečná, neboť v každém kroku sestupujeme o jednu úroveň haldy níže a úrovní máme pouze konečně mnoho. V nejhorším případě navíc ještě voláme funkci balance\_up, která je z konečnosti heapify\_up také konečná. Celkově je proto delete\_min konečná.

Zřejmě pokud $ |A| = 0 $, pak také $ |B| = 0 $ a nemáme co mazat, tedy ani vrátit,
tedy skončíme s chybou. Pokud je splněna podmínka řádku 4, pak mámě ve struktuře
jen jeden prvek a ten pop\_tail korektně odstraní a vrátí. Dále díky podmínce 1 je
každý prvek haldy $B $ větší roven nějakému prvku z $ A $ a proto minimum se nutně musí nacházet právě v $ A $. Protože $ A $ je navíc minimová halda, bude minimum právě $ A_0 $. \\

Použití funkce pop\_tail nám zajistí, že struktura zůstane vyvážená - podmínka 1. Rekurzivním voláním heapify\_down zajistíme, že A bude opět minimovou haldou.
Dá se nahlédnout, že heapify\_down vlastně jen
posune nějakou cestu $[A'_{i_0}, \dots, A'_{i_m}]$ o jednu úroveň nahoru. Pak ale pro odpovídající cestu $[B_{i_0}, \dots, B_{i_m}]$ platí 
\begin{equation}\label{bagr}
 B_{i_j} \geq B_{i_{j+1}} \geq A'_{i_{j + 1}} = A_{i_j}.
\end{equation}
Toto má smysl vždy když je definováno $ B_{i_{j+1}} $ a $ A'_{i_{j + 1}} $. 

Rekurze se zastaví v jednom ze tří případů:
\begin{itemize}
\item left = Nil (V tomto případě nemá aktuální uzel žádné potomky - ploché dno):  \\
Nyní z výše uvedené argumentace plyne, že podmínka 2 je splněna pro všechny prvky, až na ten aktuální, který nemá potomky. Pak buďto aktuální prvek z A nemá protějšek v B a jsme hotovi. Jinak v případě, že je podmínka 2 pro tyto dva prvky porušena, tak ji prohozením prvků opravíme a pomocí stejné argumentace jako v případě insertu se ukáže, že dvojí zavolání heapify\_up zajistí splění zbývajících podmínek.
\item right = Nil (Jeden potomek - schod):  \\
Pak pokud je to třeba, opravíme aktuální prvek abych zajistili, že A bude halda. Následně použitím balance\_up zajistíme splnění všech ostatních podmínek jako v předchozím případě.
\item $ X = \emptyset $ (potomci jsou větší než aktuální uzel): \\
Protože pro každou dvojici prvků na cestě v $ A $ a $ B $ po aktuální uzel máme dobře definovaný alespoň jeden pár odpovídajících si potomků, plyne z \eqref{bagr} platnost podmínky 2 a také všech ostatních podmínek invariantu. 

\end{itemize}
\end{proof}

\end{document}