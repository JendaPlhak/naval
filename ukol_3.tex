\documentclass[12pt,a4paper]{article}
\usepackage[utf8]{inputenc}
\usepackage[T1]{fontenc}
\usepackage[czech]{babel}
\usepackage{a4wide}
\usepackage{amsmath, amsthm, amsfonts, amssymb, graphicx, url, fancyhdr,multicol,enumerate,titling}
\usepackage{algpseudocode}
\usepackage[]{algorithm2e}
\usepackage{listings}
\theoremstyle{plain}
\newtheorem{thm}{Theorem} %[section]
\newtheorem{lemma}[thm]{Lemma}

\lstdefinelanguage{algpseudocode}{
  keywordstyle=[1]{\keywordstyle},
  keywordstyle=[2]{\operatorstyle},
  keywordstyle=[3]{\typestyle},
  keywordstyle=[4]{\functionstyle},
  identifierstyle={\identifierstyle},
  keywords=[1]{%
    begin,end,%
    program,procedure,function,subroutine,%
    while,do,for,to,next,repeat,until,loop,continue,endwhile,endfor,endloop,%
    if,then,else,endif,%
    return},
  literate={-}{$-$}1 {^}{$^\wedge$}1
           {>}{{$>$\ }}1 {<}{{$<$\ }}1
           {>=}{{$\geqslant$\ }}1 {<=}{{$\leqslant$\ }}1
           {:=}{{$\gets$\ }}1 {!=}{{$\ne$\ }}1 {<>}{{$\ne$\ }}1
           {->}{{$\;\to\;$}}1
           {&&}{{\keywordstyle and\ }}4 {{||}}{{\keywordstyle or\ }}3
           {;}{\hspace{0.2em};}2 {,}{\hspace{0.2em},}2,
}

\lstset{%
  language={algpseudocode},
  columns=fullflexible,
  numbers=left,
  numberstyle=\scriptsize,
}

\newcommand\keywordstyle{\rmfamily\bfseries\upshape}
\newcommand\operatorstyle{\rmfamily\mdseries\upshape}
\newcommand\typestyle{\rmfamily\mdseries\upshape}
\newcommand\functionstyle{\rmfamily\mdseries\scshape}

\newcommand\identifierstyle{\rmfamily\mdseries\itshape}

\newcommand\addkeywords[1]{%
  \lstset{morekeywords=[1]{#1}}}

\newcommand\addoperators[1]{%
  \lstset{morekeywords=[2]{#1}}}

\newcommand\addtypes[1]{%
  \lstset{morekeywords=[3]{#1}}}

\newcommand\addfunctions[1]{%
  \lstset{morekeywords=[4]{#1}}}

\newcommand{\Z}{\mathbb{Z}}
\newcommand{\N}{\mathbb{N}}
\newcommand{\It}{\operatorname{It}}
\newcommand{\ord}{\operatorname{ord}}

\begin{document}
\pagestyle{fancy}                      %Pro větší­ možnosti práce se záhlaví­mi a zápatími
\fancyhf{}                             %"vyčištění záhlaví a zápatí"
%\renewcommand{\headheight}{25 pt}                  %
\addtolength{\topmargin}{-30 pt}                   %
\setlength{\headsep}{10 pt}                      %
\fancyhead[L]{{\emph{IV003 - sada 3, příklad 3}}}  %
\fancyhead[R]{{\emph{Jan Plhák (UČO 408420), Vladimír Sedláček (UČO 408178)}}}                  % Nastavení­ pro titulní­ stranu
%\fancyfoot[L]{Školní rok 2009/2010}                %
%\renewcommand{\footrulewidth}{0.8 pt}              %
\renewcommand{\headrulewidth}{1 pt}                %               %

\addfunctions{is_interleave}

V následujícím kódu budeme předpokládat, že zaindexování do proměnných - řetězců $pat1 $ a $ pat2 $ je cyklické, tedy $ pat1[i] := pat1[i \operatorname{\%} |pat1|] $ (podobně pro $pat2$), což se dá implementačně velmi snadno zařídit.


\begin{lstlisting}[mathescape]
function is_interleave(str, pat1, pat2)
  if $|pat1| = |pat2| = 0$
    return $|str| = 0$
  if $|pat1| = 0$
    swap($pat1, pat2$)
  if $|pat2| = 0$
    for $ 0 \leq i < |str|$
      if $str[i] \neq pat1[i]$
        return false
    return true
    
  s := $|str| + 1$
  IL := array of size $ s \times s $, all entries set $\textit{to}$ false
  IL[0][0] := true
  
  for $ 1 \leq i < s $
    IL[0][i] := IL[0][i-1] && pat2[i-1] = str[i-1]
    IL[i][0] := IL[i-1][0] && pat1[i-1] = str[i-1]
    
  for $ 1 \leq i < s $ 
    for $ 1 \leq j < s - i $ 
      c := str[i + j - 1]
      IL[i][j] := (IL[i-1][j] && pat1[i-1] = c) || (IL[i][j-1] && pat2[j-1] = c)
      
  for $0 \leq i < s$
    if IL[i][size - 1 - i]
      return true;
  return false;
\end{lstlisting}

Vyšetříme korektnost uvedeného algoritmu. Z omezené délky všecy cyklů a nepřítomnosti rekurze plyne, že výpočet bude vždy konečný.

Dále postupujme postupně a zkoumejme jednotlivé větve algoritmu. Pokud jsou oba vzory $ pat1 $ i $ pat2 $ prázdné řetězce, bude $ str $ jejich spojením jen pokud je sám také prázdný řetězec. Dále předpokládáme, že $ pat1 $ je neprázdný (pokud by byl náhodou prázdný, provedeme na řádku 5 swap a z podmínky na řádku 2 plyne, že poté musí být neprázdný). Nyní pokud je $ pat2 $ prázdný, bude $ str $ spojením vstupních vzorů jen pokud se jedná o repetici $ pat1 $. Přesně to vyšetří řádky 7-10. 

Zbývá vyšetřit případ, kdy $ |pat1| \neq 0$ a $ |pat2| \neq 0 $. Tuto situaci řeší řádky 12-28. Pro pole $ IL $ dokážeme, že po tom, co navštívíme prvek $ IL[i][j] $ (na řádku 14, 17, 18 nebo 23), tak bude platit $ IL[i][j] = true \Leftrightarrow (str[0,\dots, i + j - 1]$ je prolnutím $ pat1[0,\dots,i - 1] $ a $ pat2[0,\dots,j - 1] $). Poznamenejme, že prolnutím míníme to stejné co spojením, pouze však pro řetězce konkrétní délky, bez opakování a výrazem $ pat1[0,\dots,i - 1] $ rozumíme repetici $ pat1 $ vhodné délky. Analogicky pro $ pat2 $. 
Je zřejmé, že uvedený algoritmus každý prvek $ IL[i][j] $ nastaví právě jednou a také je zřejmé, ve které části algoritmu a ve které iteraci k tomu dojde. Vyšetřeme tedy postupně jednotlivé prvky.

Pro $ IL[0][0] $ tvrzení zřejmě platí, neboť prázdný řetězec je repeticí spojením prázdných řetězců. Prvky $ IL[i][0] $ popisují situaci, kdy $ str $ je spojením prázdného řetězce a $ pat1 $. Naše ekvivalence se pro tento případ dá snadno dokázat indukcí díky tomu, že buďto je $ IL[i - 1][0] = false $ a proto jistě ani $ pat1[0,\dots,i - 1] $ není prefixem $ str $, nebo $ IL[i - 1][0] = true $ a $ pat1[0,\dots,i - 1] $ bude prefixem $ str $ právě když i aktuální znak bude odpovídat znaku v $ str $, tedy právě když $ pat1[i-1] = str[i-1] $. Analogickou argumentaci použijeme pro $ IL[0][i] $ a $ pat2 $.

\end{document}
