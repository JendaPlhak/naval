\documentclass[12pt,a4paper]{article}
\usepackage[utf8]{inputenc}
\usepackage[T1]{fontenc}
\usepackage[czech]{babel}
\usepackage{a4wide}
\usepackage{amsmath, amsthm, amsfonts, amssymb, graphicx, url, fancyhdr,multicol,enumerate,titling, tikz}
\usetikzlibrary {positioning}
%\usepackage {xcolor}
\definecolor {processblue}{cmyk}{0.96,0,0,0}
\usetikzlibrary{arrows}
\usepackage{algpseudocode}
\usepackage[]{algorithm2e}
\usepackage{listings}
\theoremstyle{plain}
\newtheorem{thm}{Theorem} %[section]
\newtheorem{lemma}[thm]{Lemma}

\lstdefinelanguage{algpseudocode}{
  keywordstyle=[1]{\keywordstyle},
  keywordstyle=[2]{\operatorstyle},
  keywordstyle=[3]{\typestyle},
  keywordstyle=[4]{\functionstyle},
  identifierstyle={\identifierstyle},
  keywords=[1]{%
    begin,end,%
    program,procedure,function,subroutine,%
    while,do,for,to,next,repeat,until,loop,continue,endwhile,endfor,endloop,%
    if,then,else,endif,%
    return},
  literate={-}{$-$}1 {^}{$^\wedge$}1
           {>}{{$>$\ }}1 {<}{{$<$\ }}1
           {>=}{{$\geqslant$\ }}1 {<=}{{$\leqslant$\ }}1
           {:=}{{$\gets$\ }}1 {!=}{{$\ne$\ }}1 {<>}{{$\ne$\ }}1
           {->}{{$\;\to\;$}}1
           {&&}{{\keywordstyle and\ }}4 {{||}}{{\keywordstyle or\ }}3
           {;}{\hspace{0.2em};}2 {,}{\hspace{0.2em},}2,
}

\lstset{%
  language={algpseudocode},
  columns=fullflexible,
  numbers=left,
  numberstyle=\scriptsize,
}

\newcommand\keywordstyle{\rmfamily\bfseries\upshape}
\newcommand\operatorstyle{\rmfamily\mdseries\upshape}
\newcommand\typestyle{\rmfamily\mdseries\upshape}
\newcommand\functionstyle{\rmfamily\mdseries\scshape}

\newcommand\identifierstyle{\rmfamily\mdseries\itshape}

\newcommand\addkeywords[1]{%
  \lstset{morekeywords=[1]{#1}}}

\newcommand\addoperators[1]{%
  \lstset{morekeywords=[2]{#1}}}

\newcommand\addtypes[1]{%
  \lstset{morekeywords=[3]{#1}}}

\newcommand\addfunctions[1]{%
  \lstset{morekeywords=[4]{#1}}}

\newcommand{\Z}{\mathbb{Z}}
\newcommand{\N}{\mathbb{N}}

\begin{document}
\pagestyle{fancy}                      %Pro větší­ možnosti práce se záhlaví­mi a zápatími
\fancyhf{}                             %"vyčištění záhlaví a zápatí"
%\renewcommand{\headheight}{25 pt}                  %
\addtolength{\topmargin}{-30 pt}                   %
\setlength{\headsep}{10 pt}                      %
\fancyhead[L]{{\emph{IV003 - sada 3, příklad 2}}}  %
\fancyhead[R]{{\emph{Jan Plhák (UČO 408420), Vladimír Sedláček (UČO 408178)}}}                  % Nastavení­ pro titulní­ stranu
%\fancyfoot[L]{Školní rok 2009/2010}                %
%\renewcommand{\footrulewidth}{0.8 pt}              %
\renewcommand{\headrulewidth}{1 pt}                %               %

\begin{enumerate}[1.]
\item
Tvrzení platí. Uvažme libovolný minimální řez $G$, jeho kapacita je rovna hodnotě maximálního toku, která je kladná. Tato kapacita je současně součtem nezáporných kapacit všech hran daného řezu, takže kapacita alespoň jedné z těchto hran musí být také kladná. Po jejím odstranění se hodnota minimálního řezu sníží, takže se sníží i hodnota maximálního toku.\qed
    
\item 
Tvrzení neplatí. Uvažme následující protipříklad:
\begin {center}
    \begin {tikzpicture}[-latex ,auto ,node distance =3 cm and 4cm ,on grid ,
    semithick ,
    state/.style ={ circle ,top color =white , bottom color = processblue!20 ,
    draw,processblue , text=blue , minimum width =1 cm}]
    \node[state] (D)
    {$w$};
    \node[state] (A) [above left=of D] {$s$};
    \node[state] (C) [above right =of D] {$v$};
    \node[state] (B) [above right =of A] {$u$};
    \node[state] (E) [ right =of C] {$t$};

    \path (A) edge [bend right = 0] node[below =0.15 cm] {$1/1$} (D);
    \path (A) edge [bend left =0] node[above] {$1/1$} (B);
    \path (B) edge [bend right = -0] node[above =0.15 cm] {$1/1$} (C);
    \path (D) edge [bend left =0] node[below =0.15 cm] {$1/1$} (C);
    \path (C) edge [bend left =0] node[below =0.15 cm] {$2/10$} (E);
    \end{tikzpicture}
\end{center}
Znázorněný tok je zřejmě maximální a má hodnotu $2$. Přitom po odstranění libovolné z hran $(s,u),(s,w),(u,v),(w,v)$ se jeho hodnota sníží o jedna a při odstranění hrany $(v,t)$ se jeho hodnota sníží o dva, proto je hrana $(v,t)$ nejdůležitější hranou této sítě. Minimálním řezem je však například dvojice hran $((s,u),(s,w))$ (případně také $((s,u),(w,v))$ nebo $((u,v),(s,w))$ nebo $((u,v),(w,v))$), takže hrana, která má v minimálním řezu nejvyšší kapacitu, je některá ze hran s kapacitou $1$, tedy nikoli hrana $(v,t)$, která je nejdůležitější hranou této sítě.\\
(Pokud by vadilo, že po odstranění hrany $(v,t)$ bude graf nespojitý, mohli bychom tento protipříklad modifikovat přidáním hrany $(s,t)$ s kapacitou $\frac{1}{2}$ a argumentovat zcela analogicky).
\qed
\end{enumerate}

\end{document}