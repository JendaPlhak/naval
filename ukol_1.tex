\documentclass[12pt,a4paper]{article}
\usepackage[utf8]{inputenc}
\usepackage[T1]{fontenc}
\usepackage[czech]{babel}
\usepackage{a4wide}
\usepackage{amsmath, amsthm, amsfonts, amssymb, graphicx, url, fancyhdr,multicol,enumerate,titling}
\usepackage{algpseudocode}
\usepackage[]{algorithm2e}
\usepackage{listings}
\theoremstyle{plain}
\newtheorem{thm}{Theorem} %[section]
\newtheorem{lemma}[thm]{Lemma}

\lstdefinelanguage{algpseudocode}{
  keywordstyle=[1]{\keywordstyle},
  keywordstyle=[2]{\operatorstyle},
  keywordstyle=[3]{\typestyle},
  keywordstyle=[4]{\functionstyle},
  identifierstyle={\identifierstyle},
  keywords=[1]{%
    begin,end,%
    program,procedure,function,subroutine,%
    while,do,for,to,next,repeat,until,loop,continue,endwhile,endfor,endloop,%
    if,then,else,endif,%
    return},
  literate={-}{$-$}1 {^}{$^\wedge$}1
           {>}{{$>$\ }}1 {<}{{$<$\ }}1
           {>=}{{$\geqslant$\ }}1 {<=}{{$\leqslant$\ }}1
           {:=}{{$\gets$\ }}1 {!=}{{$\ne$\ }}1 {<>}{{$\ne$\ }}1
           {->}{{$\;\to\;$}}1
           {&&}{{\keywordstyle and\ }}4 {{||}}{{\keywordstyle or\ }}3
           {;}{\hspace{0.2em};}2 {,}{\hspace{0.2em},}2,
}

\lstset{%
  language={algpseudocode},
  columns=fullflexible,
  numbers=left,
  numberstyle=\scriptsize,
}

\newcommand\keywordstyle{\rmfamily\bfseries\upshape}
\newcommand\operatorstyle{\rmfamily\mdseries\upshape}
\newcommand\typestyle{\rmfamily\mdseries\upshape}
\newcommand\functionstyle{\rmfamily\mdseries\scshape}

\newcommand\identifierstyle{\rmfamily\mdseries\itshape}

\newcommand\addkeywords[1]{%
  \lstset{morekeywords=[1]{#1}}}

\newcommand\addoperators[1]{%
  \lstset{morekeywords=[2]{#1}}}

\newcommand\addtypes[1]{%
  \lstset{morekeywords=[3]{#1}}}

\newcommand\addfunctions[1]{%
  \lstset{morekeywords=[4]{#1}}}

\newcommand{\Z}{\mathbb{Z}}
\newcommand{\N}{\mathbb{N}}

\begin{document}
\pagestyle{fancy}                      %Pro větší­ možnosti práce se záhlaví­mi a zápatími
\fancyhf{}                             %"vyčištění záhlaví a zápatí"
%\renewcommand{\headheight}{25 pt}                  %
\addtolength{\topmargin}{-30 pt}                   %
\setlength{\headsep}{10 pt}                      %
\fancyhead[L]{{\emph{IV003 - sada 2, příklad 1}}}  %
\fancyhead[R]{{\emph{Jan Plhák (UČO 408420), Vladimír Sedláček (UČO 408178)}}}                  % Nastavení­ pro titulní­ stranu
%\fancyfoot[L]{Školní rok 2009/2010}                %
%\renewcommand{\footrulewidth}{0.8 pt}              %
\renewcommand{\headrulewidth}{1 pt}                %               %

\begin{enumerate}
\item Na protipříkladu ukážeme, že zadaný algoritmus není korektní. Uvažme množinu $A=\{-\frac{5}{4},-\frac{1}{2},-\frac{1}{4},\frac{1}{4},\frac{1}{2},\frac{5}{4}\}$. Pak (prostým projitím možností) platí, že interval $[-\frac{1}{2},\frac{1}{2}]$ pokrývá největší počet bodů z $A$ ze všech jednotkových intervalů. Algoritmus ho proto přidá do $S$ a z $A$ se stane množina $\{-\frac{5}{4},\frac{5}{4}\}$. Tyto dva body však nelze pokrýt jediným jednotkovým intervalem, proto je daný algoritmus postupně ve dvou krocích pokryje dvěma různými intervaly (navíc žádný z nich zřejmě nebude $[-\frac{1}{2},\frac{1}{2}]$). Výstupem algoritmu tedy bude množina $S$, která bude obsahovat tři různé jednotkové intervaly. Původní množina $A$ ovšem lze pokrýt intervaly $[-\frac{5}{4},-\frac{1}{4}]$ a $[\frac{1}{4},\frac{5}{4}]$. Zadaný algoritmus tedy vrátil tři intervaly místo optimálních dvou, takže není korektní.

\item Ukážeme, že zadaný algoritmus je korektní (s tím, že všechny použité intervaly považujeme za uzavřené). Nechť na konci běhu algoritmu množina $S$ obsahuje intervaly $[a_{j_1},a_{j_1}+1],\dots,[a_{j_k},a_{j_k}+1]$. Potom z konstrukce algoritmu plyne, že pro libovolné $l,m\in\{1,\dots,k\},l\neq m$ platí $|a_{j_l}-a_{j_m}|>1$ (pokud by platilo $|a_{j_l}-a_{j_m}|\leq 1$ a např. $a_{j_l}\leq a_{j_m}$, byl by bod $a_{j_m}$ pokryt intervalem $[a_{j_l},a_{j_l}+1]$, takže by v průběhu algoritmu byl odstraněn z $A$ ve stejném kroku jako $a_{j_l}$, tudíž by se nikdy nemohl stát nejmenším aktuálním bodem $S$ a nemohl by tak nakonec být krajním bodem intervalu z $S$). Jinými slovy $A$ obsahuje $k$ bodů, z nichž žádné dva nelze současně pokrýt jednotkovým intervalem. Proto není možné $A$ pokrýt méně než $k$ intervaly, takže zadaný algoritmus, který ho pokryl $k$ intervaly, našel nejlepší řešení, a tedy je korektní. Jeho časová složitost je zřejmě $\mathcal{O}(n)$, protože na každý bod se díváme právě dvakrát (poprvé ho buď používáme jako krajní bod aktuálního intervalu, nebo testujeme, zda do aktuálního intervalu patří; a podruhé ho odstraňujeme z $A$) a do $S$ celkově přidáme nejvýše $n$ intervalů.
\end{enumerate}

\end{document}